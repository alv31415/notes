\documentclass{exam}
\usepackage[utf8]{inputenc}

\usepackage{mynotes}

\title{Honours Algebra - Week 5 - Equivalence Relations, The First Isomorphism Theorem \& Modules}
\author{Antonio León Villares}
\date{February 2022}

\begin{document}

\maketitle

\tableofcontents

\pagebreak

\textit{Based on the notes by Iain Gordon, Sections 3.5 - 3.7}

\section{Equivalence Relations}

\subsection{Defining Equivalence Relations}

\begin{itemize}
    \item \textbf{What is a relation?}
    \begin{itemize}
        \item a \textbf{relation} $R$ on a set $X$ is a \textbf{subset}:
        \[
        R \subset X \times X
        \]
        \item we describe an element $(x,y) \in R$ via:
        \[
        xRy
        \]
    \end{itemize}
    \item \textbf{What is an equivalence relation?}
    \begin{itemize}
        \item a \textbf{relation}, typically denote $\sim$, satsifying:
        \begin{enumerate}
            \item \textbf{Reflexivity}
            \[
            x \sim x
            \]
            \item \textbf{Symmetry}
            \[
            x \sim y \ \iff \ y \sim x
            \]
            \item \textbf{Transitivity}
            \[
            x \sim y \ \wedge \ y \sim z \ \implies \ x \sim z
            \]
        \end{enumerate}
    \end{itemize}
\end{itemize}

\subsubsection{Examples}

\begin{itemize}
    \item simple equivalence relations include:
    \[
    x \sim y \ \iff \ x = y \qquad x \sim y \ \iff \ x^2 = y^2
    \]
    The first one is more ``restrictive", since in the second one tuples like $(x,-y)$ and $(-x,y)$ are allowed.
    \item \textbf{congruence modulo $m$} also defines an equivalence relation:
    \[
    x \sim y \ \iff \ \modulom{x}{y}
    \]
    \begin{itemize}
        \item $\modulom{x}{x}$ (reflexivity)
        \item $\modulom{x}{y} \ \iff \ \modulom{y}{x}$ (symmetry)
        \item $\modulom{x}{y} \ \wedge \ \modulom{y}{z} \ \implies \ \modulom{x}{z}$ (transitivity)
    \end{itemize}
    \item a more interesting example is that of matrix conjugacy:
    \[
    A \sim B \ \iff \ \exists X : B = XAX^{-1}, \qquad A,X,B \in Mat(n;F)
    \]
    \begin{itemize}
        \item $IAI^{-1} = A \ \implies \ A \sim A$
        \item $B = XAX^{-1} \ \implies \ A = YBY^{-1}, \qquad Y = X^{-1}$
        \item $B = XAX^{-1}, C = YBY^{-1} \ \implies \ C = ZAZ^{-1}, \qquad Z = YX$
    \end{itemize}
    This relates to the notion of \textbf{similar matrices}, discussed in W2, whereby basis matrices were similar. That is, if $N = \cript{f}{B}{B}$ and $N = \cript{f}{A}{A}$, then $N,M$ are similar in the sense that with $T = \cript{id_V}{A}{B}$:
    \[
    N = T^{-1}MT
    \]
\end{itemize}

\subsubsection{Exercises (TODO)}

\begin{questions}

\question \textbf{Show that the relation $\sim$ on $Mat(n \times m; F)$, defined by:
\[
A \sim B \ \iff \ \exists P \in GL(n;F), Q \in GL(m;F) : B = PAQ 
\]
is an equivalence relation.}

\question \textbf{Show that isomorphism is an equivalence relation on finite dimensional vector spaces over a field $F$.}

\end{questions}

\subsection{Equivalence Classes}

\begin{itemize}
    \item \textbf{What is an equivalence class?}
    \begin{itemize}
        \item consider a set $X$ with \textbf{equivalence relation} $\sim$
        \item an \textbf{equivalence class} for $x \in E$ is a subset $E \subseteq X$ such that:
        \[
        E(x) := \{z \ | \ x \sim z, \ x \in X\}
        \]
    \end{itemize}
    \item \textbf{What is a representative of an equivalence class?}
    \begin{itemize}
        \item an element $e \in E(x)$
    \end{itemize}
    \item \textbf{What is a system of representatives?}
    \begin{itemize}
        \item a subset $Z \subseteq X$
        \item it contains excatly \textbf{one} element from each \textbf{equivalence class} $E(x), x \in X$
    \end{itemize}
    \item \textbf{What are some properties of equivalence classes?}
    \begin{itemize}
        \item the following notions are \textbf{equivalent}:
        \begin{enumerate}
            \item $x \sim y$
            \item $E(x) = E(y)$ (this follows from (1) + symmetry)
            \item $E(x) \cap E(y) \neq \emptyset$ (this follows from (1) + reflexivity, which means that $x \in E(x), x \in E(y)$)
        \end{enumerate}
    \end{itemize}
\end{itemize}

\subsubsection{Examples}

\begin{itemize}
    \item if $X$ is a set of students, with equivalence relation ``same degree", each equivalence class contains all students which prusue the same degree
    \item if $X = \mathbb{R}$, then the equivalence relation:
    \[
    x \sim y \ \iff \ x - y \in \mathbb{Z}
    \]
    has equivalence classes like:
    \[
    E(1.2) = \{\ldots, -2.8, -1.8, -0.8, 0.2, 1.2, 2.2, \ldots\}
    \]
    More generally, if $z \in \mathbb{Z}$, the equivalence relation tells us that:
    \[
    y \in E(x) \ \implies \ y - x = z \ \implies \ y = x + z \ \implies \ E(x) = \{x + z \ | \ x \in X, z \in \mathbb{Z}\}
    \]
    (here we have used symmetry)
    \item if we define:
    \[
    x \sim y \ \iff \ \modulom{x}{y}
    \]
    then the equivalence classes are familiar:
    \[
    E(x) = \bar{x} = \{x + mz \ | \ z \in \mathbb{Z}\}
    \]
    Furthermore, if $m > 0$, a \textbf{system of representatives} will contain an element of each \textbf{equivalence class}, which can be:
    \[
    \{0,1,2,\ldots,m-1\}
    \]
    where $0 \in E(0), 1 \in E(1)$, and so on. However, in general, we can pick:
    \[
    \{a, a+1, a+2, \ldots, a+m-1\}
    \]
    where $a \in \mathbb{Z}$
\end{itemize}

\subsubsection{Exercises (TODO)}

\begin{questions}

\question \textbf{Show that the $n \times m$ matrices over a field $F$ in Smith-Normal Form form a system of representatives for the equivalence relation:
\[
A \sim B \ \iff \ \exists P \in GL(n;F), Q \in GL(m;F) : B = PAQ 
\]}

\question \textbf{Show that the set:
\[
\{F^n \ | \ n \in \mathbb{Z}_{\geq 0}\}
\]
is a system of representatives for the equivalence relation defined by an
isomorphism on finite dimensional vector spaces over a field $F$. Show that another system of representatives for this equivalence relation is:
\[
\{F[X]_{<n} \ | \ n \in \mathbb{Z}_{\geq 0}\}
\]}

\end{questions}

\subsection{The Set of Equivalence Classes}

\begin{itemize}
    \item \textbf{What is the set of equivalence classes?}
    \begin{itemize}
        \item let $X$ be a set, with equivalence relation $\sim$
        \item the \textbf{set of equivalence classes} is a \textbf{subset} of the \textbf{power set} $\mathcal{P}(X)$
        \item it is the set containing all equivalence classes of $X$:
        \[
        (X / \sim) := \{E(x) \ | \ x \in X\}
        \]
        \item this is also known as the \textbf{quotient set}
    \end{itemize}
    \item \textbf{What canonical mapping arises from this definition?}
    \textbox{A \textbf{canonical map} is a map or morphism between objects that arises naturally from the \textbf{definition} or the \textbf{construction} of the objects. In general, it is the map which preserves the widest amount of structure, and it tends to be unique.}
    \begin{itemize}
        \item in this case, a \textbf{canonical map} is of the form:
        \[
        can : X \to (X / \sim)
        \]
        \[
        can(x) = E(x)
        \]
        \item this is a \textbf{surjection}, since each equivalence class $E(x)$ contains at least one element in $X$ (so ``worst case", each $x \in X$ maps to a unique $E(x)$)
    \end{itemize}
\end{itemize}

\subsubsection{Examples: Canonical Mappings Preserving Structure (as Homomorphisms)}

\begin{enumerate}
    \item \textbf{Abelian Groups}
    \begin{itemize}
        \item $A$ is an \textbf{abelian group}; $B$ is a \textbf{subgroup} of $A$
        \item define an equivalence relation.
        \[
        x \sim y \ \iff \ x - y \in B
        \]
        \item the equivalence classes are:
        \[
        y - x = b \in B \ \implies \ E(x) = \{x + b \ | \ b \in B\}
        \]
        (here w use the fact that the group is abelian, so $x + b = b+x$)
        \item the \textbf{quotient set} is $A/B \equiv A/\sim$, an \textbf{abelian group}, defined by:
        \[
        E(x) + E(y) = E(x + y) = \{x + y + b \ | \ b \in B\}
        \]
        \item the canonical mapping $can : A \to A/B$ is a \textbf{surjective homomorphism}, with kernel being $B$ (since $0 \in A/B = \{0 + b \ | \ b \in B\} = B$, and any element $b \in B$ will get mapped to this set)
        \item $A / B$ is the \textbf{quotient abelian group} of $A$ by the subgroup $B$
    \end{itemize}
    \item \textbf{Non-Abelian Group}
    \begin{itemize}
        \item $G$ is a \textbf{group}, and $H$ is a \textbf{normal subgroup}: that is, if $h \in H$ and $g \in G$, then:
        \[
        ghg^{-1} \in H
        \]
        \item define an equivalence relation:
        \[
        x \sim y \ \iff \ xy^{-1} \in H
        \]
        \item the equivalence classes are given by the \textbf{left} and \textbf{right} cosets:
        \[
        E(x) = xH = Hx \subseteq G
        \]
        This is because if $xy^{-1} \in H$, by symmetry, $yx^{-1} = h \in H$, so $y = hx$, meaning that $E(x) = \{hx \ | \ h \in H\}$. the equivalence relation is defined by $$xy^{-1} \in H$$. Moreover, since $G$ is a group, and $H$ is a normal subgroup, we know that $g^{-1}hg \in H$. Thus, if $xy^{-1} \in H$, we must also have $y^{-1}(xy^{-1})y = y^{-1}x \in H$. Hence, $y^{-1} \sim x^{-1}$, and again by symmetry, $x^{-1} \sim y^{-1} \ \implies \ x^{-1}y \in H \ \implies \ y = xh \ \implies \ E(x) = xH$.
        \item the quotient set $G / \sim \equiv G / H$ is the group with operations:
        \[
        E(x)E(y) = E(xy)
        \]
        \item the canonical \textbf{surjective homomorphism} $can : G \to G / H$ has kernel $H$, since $hH = Hh = H$, so $\forall h \in H, can(h) = H$, and $H$ is the identity element in $G/H$
        \item this relates to \textbf{Lagrange's Theorem}, which states that:
        \[
        |G| = |G/H||H|
        \]
        which is proved by noting that each coset $E(x)$ has exactly $|H|$ elements (since it is given by $xH = Hx$), and that $G$ is the
        disjoint union of $|G/H|$ cosets (the union is disjoint because otherwise we'd have elements belonging to more than 1 equivalence classes; there are $|G/H|$ cosets because $G/H$ is the set of all cosets).
        \item \item $G / H$ is the \textbf{quotient group} of $G$ by the normal subgroup $H$
    \end{itemize}
    \item \textbf{F-Vector Spaces}
    \begin{itemize}
        \item let $V$ be a \textbf{F-Vector Space}, with subspace $W$
        \item define an equivalence relation:
        \[
        x \sim y \ \iff \ x - y \in W
        \]
        \item as in the first case, the equivalence classes are:
        \[
        y - x = w \ \implies \ E(x) = \{x + w \ | \ w \in W\} = x + W
        \]
        \item the \textbf{quotient set} is an \textbf{F-Vector Space} with:
        \[
        \lambda E(x) = E(\lambda x)
        \]
        \item the canonical \textbf{surjective homomorphism} $can : V \to V/W$ has kernel $W$ (same reason as in case (1))
        \item by the exercise below, we can show that if $V$ is finite-dimensional:
        \[
        dim(V/W) = dim(V) - dim(W)
        \]
        \item $V/W$ is the \textbf{quotient vector space} of $V$ by the subspace $W$
    \end{itemize}
\end{enumerate}

\subsubsection{Examples}

\begin{itemize}
    \item if $\sim$ defines the congruence equivalence relation, modulo $m$, then:
    \[
    (\mathbb{Z} / \sim) = \mathbb{Z}_m
    \]
    This is easy to see, since as discussed above, the equivalence classes of $\sim$ are the sets $\bar{0}, \bar{1}, \ldots, \overline{m-1}$, and the set of all these elements is precisely $\mathbb{Z}_m$
\end{itemize}

\subsubsection{Exercises (TODO)}

\begin{questions}

\question \textbf{Let $R = F$ be a field, $V$ and $F$-vector space, and $W \subseteq V$ a subspace of $V$. The quotient $V/W$ is the \textit{quotient vector space}, and $can : V \to V/W$ is a linear mapping.
Assume that $dim V = m < \infty$. By the \textit{Dimension Estimate for Vector Subspaces}, $dim W = n \leq m$. Let:
\[
\{\vec{v}_1, \ldots, \vec{v}_n\}
\]
be a basis for $W$. Using the \textit{Steinitz Exchange Theorem}, we can extend it to a basis of $V$:
\[
\{\vec{v}_1, \ldots, \vec{v}_n, \vec{v}_{n+1}, \ldots, \vec{v}_m\}
\]
Show that:
\[
\{\vec{v}_{n+1} + W, \ldots, \vec{v}_m + W\}
\]
is a basis for the vector space $V / W$. Hence, deduce that:
\[
dim(V /W) = dim V - dim W
\]
}

\end{questions}

\subsection{Remark: A Very Important Remark At That}\label{vir}

\textbox{
Consider $\sim$ as an equivalence relation on $X$, and let $f : X \to Z$ be a mapping, such that:
\[
x \sim y \ \implies \ f(x) = f(y)
\]
In other words, whatever the equivalence relation is, it is such that all elements in the same \textbf{equivalence class} are mapped to the same value under $f$
\\
Then, there exists a \textbf{unique} mapping:
\[
\bar{f} : (X/\sim) \to Z
\]
This mapping is simple to define:
\[
\bar{f}(E(x)) = f(x)
\]
such that:
\[
f = \bar{f} \circ can
\]
This can be summarised by the following diagram:
\pic{tri.png}{0.5}
This is known as the \textbf{universal property of the set of equivalence classes}
}
\textbox{
A more interesting case occurs when
\[
f : X \to Z
\]
is \textbf{any} mapping, and we define:
\[
x \sim y \ \iff \ f(x) = f(y)
\]
Then, we will have that:
\[
\bar{f} : (X / \sim) \to im f
\]
is a \textbf{bijection}. This bijection is a prelude to the \textbf{First Isomorphism Theorem}.
}

\subsection{A Well-Defined Mapping}

\begin{itemize}
    \item \textbf{When is a mapping well-defined?}
    \begin{itemize}
        \item consider a mapping:
        \[
        g : (X/\sim) \to Z
        \]
        \item $g$ is \textbf{well-defined} if there exists a mapping:
        \[
        f : X \to Z
        \]
        such that:
        \[
        x \sim y \ \implies \ f(x) = f(y)
        \]
        \item here we recognise $g = \bar{f}$
    \end{itemize}
    \item \textbf{Why are well-defined mappings important?}
    \begin{itemize}
        \item they \textbf{solidify} the notion of \textbf{equivalence}
        \item they ensure that elements in the \textbf{same equivalence class} are mapped to the \textbf{same} value
        \item this means that equivalent elements in $X$ are the same in $Z$
        \item more on this in \href{https://proofwiki.org/wiki/Definition:Well-Defined/Mapping}{Proof-Wiki}
    \end{itemize}
\end{itemize}

\subsubsection{Examples}

\begin{itemize}
    \item recall the equivalence relation:
    \[
    a \sim b \ \iff \ a - b \in \mathbb{Z}
    \]
    with equivalence classes:
    \[
    E(a) = \{\ldots, a-2, a-1, a, a+1, a+2, \ldots \}
    \]
    Further consider:
    \[
    f : \mathbb{R} \to \mathbb{R} \qquad f(x) = \cos(x)
    \]
    \[
    g : \mathbb{R} \to \mathbb{R} \qquad f(x) = \cos(2\pi x)
    \]
    Then, $f$ is \textbf{not} well-defined, since:
    \[
    0 \sim 1 \qquad f(0) = 1 \neq f(1)
    \]
    However, $g$ \textbf{is} well-defined. Indeed:
    \[
    a \sim b \ \implies \ a = b+z, z \in \mathbb{Z}
    \]
    So:
    \[
    g(a) = \cos(2\pi a) = \cos(2\pi b + 2\pi z) = \cos(2\pi b) = g(b)
    \]
    where we exploit the fact that $\cos$ is $2\pi$ periodic.
    \smallskip 
    Moreover, we can define:
    \[
    \bar{g} : (\mathbb{R} / \sim) \to \mathbb{R}
    \]
    via:
    \[
    \bar{g}(E(a)) = g(a) = cos(2\pi a)
    \]
\end{itemize}

\subsubsection{Exercises (TODO)}

\begin{questions}

\question \textbf{Define a relation $\sim$ on $\mathbb{X} \times \mathbb{N}$ by:
\[
(x,y) \sim (a,b) \ \iff \ x+b = y+a
\]
}

\begin{parts}
\part 
Show that $\sim$ is an equivalence relation

\part Let $\bar{\mathbb{N}} = (\mathbb{N} \times \mathbb{N}/\sim)$. Show that addition on $\mathbb{N}$ induces a \textit{well-defined} addition on $\bar{\mathbb{N}}$

\part Show that with this addition, $\bar{\mathbb{N}}$ is an abelian group

\part Show that
\[
nat : \mathbb{N} \to \bar{\mathbb{N}} 
\]
is an additive mapping, where:
\[
nat(a) = E((a+n, n)), \forall n \in \mathbb{N}
\]
That is:
\[
nat(a+b) = nat(a) + nat(b)
\]

\part Show that $\bar{\mathbb{N}}$ is \textit{isomorphic} as a group to $(\mathbb{Z}, +)$

\end{parts}

\end{questions}

\section{Factor Rings}

\subsection{Motivation 1: Equivalence Relations From Kernels}

\textbox{
We just showed that \textbf{mappings between sets} generate \textbf{equivalence relations}. In particular, consider a \textbf{ring homomorphism}:
\[
f : R \to S
\]
We can define an \textbf{equivalence relation} on $R$:
\[
x \sim y \ \iff \ f(x) = f(y)
\]
By properties of homomorphism:
\begin{align*}
    x \sim y \ \iff \ &f(x) = f(y) \\
    \iff \ &f(x) - f(y) = 0_S \\
    \iff \ &f(x - y) = 0_S \\
    \iff \ &x - y \in ker(f)\\
\end{align*}
This then defines the \textbf{equivalence classes} via:
\[
y - x = k \in ker(f) \ \implies \ y = x + k
\]
so in particular:
\[
E(x) = x + ker(f) = \{x + k \ | \ k \in ker(f)\}
\]
}

\pagebreak

\subsection{Motivation 2: Equivalence Relations From Ideals}

In fact, all the above generalises easily to \textbf{ideals}:

\textbox{
If $I$ is an ideal of a ring $R$, and $f : R \to S$, then the following is an equivalence relation:
\[
r_1 \sim r_2 \ \iff \ r_1 - r_2 \in I
\]
\begin{enumerate}
    \item $r_1 - r_1 = 0_R \in I \ \iff \ r_1 \sim r_1$ (since $0$ is always part of an ideal). Hence, \textbf{reflexivity} holds.
    \item if $r_1 \sim r_2$, then:
    \[
    r_1 - r_2 \in I
    \]
    Since ideals are closed under substraction, it follows that:
    \[
    -(r_1 - r_2) = r_2 - r_1 \in I
    \]
    so we have $r_2 \sim r_1$. Thus, \textbf{symmetry} holds.
    \item if $r_1 \sim r_2$ and $r_2 \sim r_3$, then:
    \[
    r_1 - r_2 \in I
    \]
    \[
    r_2 - r_3 \in I
    \]
    Ideals are closed under substraction/addition, so:
    \[
    (r_1 - r_2) + (r_2 - r_3) = r_1 - r_3 \in I
    \]
    so we have $r_1 \sim r_3$. Thus, \textbf{transitivity} holds.
\end{enumerate}
As above, the equivalence classes then become:
\[
E(r_1) = r_1 + I = \{r_1 + i \ | \ i \in I\}
\]
and the \textbf{quotient of $R$ by $I$} (the set of \textbf{all equivalence classes}) is:
\[
R / I
\]
}

\pagebreak 

We have constructed $R / I$. We now go back, and discover that it is a ring.

\subsection{Motivation 3: Quotients From Ideals are Rings}

\textbox{
$R / I$ is the set of all equivalence classes, constructed from the \textbf{equivalence relation}:
\[
r_1 \sim r_2 \ \iff \ r_1 - r_2 \in I
\]
But if we think about it, this equivalence relation was originally defined as:
\[
r_1 \sim r_2 \ \iff \ f(r_1) =  f(r_2)
\]
But now recall, such an equivalence relation lead to the following bijection:
\[
\bar{f} : (R / I) \to im(f)
\]
\[
\bar{f}(E(r)) = f(r)
\]
The existence of this bijection tells us that, since $im(f)$ is a \textbf{subring} of $S$, we should expect that $R/I$ should also have a ring-like structure, since we have a one-to-one correspondance between elements in $R / S$ and a subring (in fact, if $\bar{f}$ is an \textbf{isomorphism}, $R / S$ would indeed be \textbf{isomorphic} to the subring $im(f)$).
\\
So if we have $R / I$ as a ring, we better endow it with \textbf{addition} and \textbf{multiplication}:
\[
E(r_1) + E(r_2) = E(r_1 + r_2)
\]
\[
E(r_1r_2) = E(r_1)E(r_2)
\]
This section focuses on formalising the notion of $R / I$ as a \textbf{factor ring}, defines the \textbf{Universal Property of Factor Rings} results, and has a grand finale in the \textbf{First Isomorphism Theorem}.
}

\pagebreak 

\subsection{Cosets of Rings}

\begin{itemize}
    \item \textbf{What is a coset of a ring?}
    \begin{itemize}
        \item let $R$ be a ring, and $I$ an ideal of $R$
        \item the \textbf{coset of $x$ with respect to $I$ in $R$} is the subset of $R$:
        \[
        x + I = \{x + i \ | \ i \in I\}
        \]
        \item cosets in \textbf{rings} are special cases of cosets in \textbf{groups}
    \end{itemize}
    \item \textbf{Are cosets equivalence classes?}
    \begin{itemize}
        \item as we saw above,
        \[
        x \sim y \ \iff \ x - y \in I
        \]
        defines an equivalence class:
        \[
        E(x) = x + I
        \]
        \item so \textbf{cosets} are \textbf{equivalence classes}
    \end{itemize}
    \item \textbf{Given 2 cosets, how can they be related?}
    \begin{itemize}
        \item we saw that if $x \sim y$, then:
        \[
        E(x) = E(y) \qquad E(x) \cap E(y) \neq \emptyset
        \]
        \item hence, depending on whether $x \sim y$, 2 cosets $x + I$ and $y + I$ are related in one of 2 ways:
        \begin{itemize}
            \item $x + I = y + I$
            \item or $(x + I) \cap (y+I) = \emptyset$
        \end{itemize}
    \end{itemize}
\end{itemize}

\subsection{Defining the Factor Ring}

\begin{itemize}
    \item \textbf{What is a factor ring?}
    \begin{itemize}
        \item let:
        \begin{itemize}
            \item $R$ be a ring
            \item $I$ be an ideal of $R$
            \item $\sim$ the equivalence relation on $R$:
            \[
            x \sim y \ \iff \ x - y \in I
            \]
        \end{itemize}
    \end{itemize}
    \item the \textbf{factor ring of $R$ by $I$} is nothing but the \textbf{quotient of $R$ by $I$}
    \item hence, the factor ring is $R / I$:
    \begin{itemize}
        \item the set of \textbf{equivalence classes} under $\sim$
        \item the set of \textbf{cosets} of $I$ in $R$
    \end{itemize}
\end{itemize}

\subsection{Theorem: Operations on Factor Rings}

For the \textbf{factor ring} to be a ring, we need to provide ring operations.

\textbox{
Let $R$ be a \textbf{ring}, and $I$ an \textbf{ideal}.
\\
Then, $R/I$ is a \textbf{ring}, with \textbf{addition}defined as:
\[
(x + I) \dot{+} (y + I) = (x + y) + I, \forall x,y \in R
\]
and \textbf{multiplication} defined as:
\[
(x + I) \cdot (y + I) = xy + I, \forall x,y \in R
\]
[Theorem 3.6.4]
}

For the proof we need to be careful, and show that:
\begin{itemize}
    \item $R/I$ is an \textbf{abelian} group under \textbf{addition}
    \item addition is \textbf{well-defined}
    \item $R/I$ is a \textbf{monoid} under \textbf{multiplication}
    \item multiplication is \textbf{well-defined}
    \item $R/I$ satsifies the \textbf{distributive axioms}
\end{itemize}

This proves that $R / I$ is a ring. It is important to emphasise the need to show that the operations are \textbf{well-defined}:

\textbox{
Consider $R = \mathbb{Z}$ and $I = 15\mathbb{Z} = \{15z \ | \ z \in \mathbb{Z}\}$. The \textbf{equivalence classes}, as discussed, are of the form:
\[
E(r) = \{r + 15z \ | \ z \in \mathbb{Z} \}
\]
The factor ring is our well known:
\[
\mathbb{Z}/15\mathbb{Z} = \mathbb{Z}_{15}
\]
Now, consider the following products:
\[
E(7) \cdot E(9) = E(63) \qquad  E(22) \cdot E(9) = E(198)
\]
Now, we know that 22 and 7 are congruent modulo 15, so obviously $E(7) = E(22)$ (they are the same equivalence class). The question now becomes: are $E(63)$ and $E(198)$ congruent modulo 15? How can we be sure?
\\
This is the importance of having \textbf{well-defined} operations: we are working over \textbf{equivalence classes}, so we need to ensure that any arithmetic we do doesn't depend of our choice of \textbf{representative} of the equivalence class. 
\\
In this example, any arithmetic we do shouldn't depend on the number we choose (i.e 7 and 22), but rather the \textbf{remainder} when dividing by 15.
}

\begin{proof}

\begin{enumerate}
    \item \textbf{Addition is Well-Defined}
    
    To show that addition is well-defined, we need to show that if $x,x',y,y' \in R$ and:
    \[
    E(x) = E(x') \qquad E(y) = E(y')
    \]
    (we use $E(x)$ instead of $x + I$ for ease of reading and writing)
    then:
    \[
    E(x) + E(y) = E(x') + E(y')
    \]
    Notice, by how addition is defined, this is equivalent to showing that:
    \[
    E(x + y) = E(x' + y')
    \]
    which is equivalent to showing that the two are the \textbf{same equivalence class}. In other words, we need to show that:
    \[
    (x + y) \sim (x' + y') \ \implies \ (x + y) - (x' + y') \in I
    \]
    We consider:
    \[
    (x + y) - (x' + y') = (x - x') + (y - y')
    \]
    By assumption, $E(x) = E(x')$, so $x \sim x'$, so $x - x' \in I$. Similarly, $y - y' \in I$. Since ideals are closed under addition/subtraction, it is clear that:
    \[
    (x - x') + (y - y') \in I
    \]
    Hence, addition is well-defined.
    \item \textbf{$R/I$ is an Abelian Group Under Addition}
    \begin{enumerate}
        \item \textbf{Existence of Identity}
        \[
        E(0) + E(x) = E(0 + x) = E(x) = E(x + 0) = E(x) + E(0)
        \]
        \item \textbf{Existence of Inverse}
        \[
        E(-x) + E(x) = E(-x + x) = E(0) = E(x - x) = E(x) + E(-x)
        \]
        \item \textbf{Associativity}
        \[
        (E(x) + E(y)) + E(z) = E(x + y) + E(z) = E(x + y + z) = E(x) + E(y + z) = E(x) + (E(y) + E(z))
        \]
        \item \textbf{Closure}
        This follows directly from the definition of addition.
        \item \textbf{Abelian}
        \[
        E(x) + E(y) = E(x + y) = E(y + x) = E(y) + E(x)
        \]
        (using commutativity of $R$ under addition)
    \end{enumerate}
    Hence, $R / I$ is an abealian group under addition.
    \item \textbf{Multiplication is Well-Defined}
    Again ,consider $x,x',y,y'$ with:
    \[
    E(x) = E(x') \qquad E(y) = E(y')
    \]
    we need to show that:
    \[
    E(x)E(y) = E(x')E(y')
    \]
    Since $E(x) = E(x')$, we know that:
    \[
    x \sim x' \ \implies \ x - x' \in I \ \implies \ x = x' + i, i \in I 
    \]
    Similarly:
    \[
    y = y' + j, j \in I
    \]
    Hence:
    \begin{align*}
        E(x)E(y) &= E(xy) \\
                 &= E((x' + i)(y' + j)) \\
                 &= E(x'y' + x'j + iy' + ij) \\
                 &= E(x'y') + E(x'j) + E(iy') + E(ij) \\
    \end{align*}
    Now, notice that:
\[
E(x'j) = E(iy') = E(ij) = E(0)
\]
Since $ij, iy', x'j \in I$ (since multiplying elements in $R$ by elements in $I$ produces elements in $I$), then:
\[
ij - 0 \in I \qquad iy' - 0 \in I \qquad x'j - 0 \in I
\]
So $ij \sim iy' \sim x'j \sim 0$, from which $E(x'j) = E(iy') = E(ij) = E(0)$ follows. Hence, we have shown that:
\[
E(x)E(y) = E(x'y') = E(x')E(y')
\]
so multiplication is well-defined.

\item \textbf{$R/I$ is a Monoid Under Multiplication}

\begin{enumerate}
    \item \textbf{Closure} This follows directly from the definition of multiplication.
    \item \textbf{Associativity}
    \[
    (E(x)E(y))E(z) = E(xy)E(z) = E(xyz) = E(x)E(yz) = E(x)(E(y)E(z))
    \]
    \item \textbf{Existence of Identity}
    \[
    E(x)E(1) = E(x \cdot 1) = E(x) = E(1 \cdot x) = E(1)E(x)
    \]
\end{enumerate}
Hence, $R / I$ is a monoid under multiplication.

\item \textbf{$R / I$ Satisfies Distributivity}
\[
E(x)(E(y) + E(z)) = E(x)E(y+z) = E(x(y+z)) = E(xy + xz) = E(xy) + E(xz) = E(x)E(y) + E(x)E(z)
\]
\end{enumerate}

Hence, by all of the above, $R/I$ is a ring, with \textbf{well-defined} operations.

\end{proof}

\subsubsection{Examples}

\begin{itemize}
    \item this is another way of seeing that $\mathbb{Z}/m\mathbb{Z} = \mathbb{Z}_m$ is a \textbf{ring}
    \item consider the ring $R = \F_2[X]$ (the ring of polynomials with coefficients 0 or 1), and the following 2 ideals:
    \[
    I = {}_R\langle X^2 \rangle = \{pX^2 \ | \ p \in R\}
    \]
    \[
    J = {}_R\langle X^2 + X + 1 \rangle = \{p(X^2 + X + 1) \ | \ p \in R\}
    \]
    How do we describe $R/I$ and $R/J$?
    \begin{enumerate}
        \item \textbf{Elements in the Factor Rings}
        \begin{enumerate}
            \item \textbf{Elements in $R/I$}
            For $p \in R$, We denote:
            \[
            E_I(p) = p + I
            \]
            Then, we can consider the equivalence classes, given by the relation:
            \[
            p_1 \sim p_2 \ \iff \ p_1 - p_2 \in I
            \]
            that is $p_1 \sim p_2$ if $p_1 - p_2$ has $X^2$ as a factor.
            Consider the constant polynomials first (only 2 of them)
            \[
            E_I(0) = \{i \ | \ i \in I\} \qquad E_I(1) = \{1 + i \ | \ i \in I\}
            \]
            Clearly, $0 \not\sim 1$, since $0 - 1 = 1$ (we operate in $\F_2$) which is not divisible by $X^2$. Hence, $E_I(0)$ and $E_I(1)$ are different equivalence classes.
            \bigskip
            Now considering linear polynomials:
            \[
            E_I(X) = \{X + i \ | \ i \in I\} \qquad E_I(X + 1) = \{X + 1 + i \ | \ i \in I\}
            \]
            Then:
            \begin{itemize}
                \item $X - 0 \not\in I$
                \item $X - 1 \not\in I$
                \item $(X + 1) - 0 \not\in I$
                \item $(X + 1) - 1 \not\in I$
                \item $(X + 1) - X \not\in I$
            \end{itemize}
            Hence, we have 2 more equivalence classes:
            \[
            E_I(X) \qquad E_I(X + 1)
            \]
            Now, consider a polynomial $p \in R$ with $deg(p) \geq 2$. We can factorise it as:
            \[
            p(X) = q(X)(X^2) + r(X)
            \]
            such that $deg(r) < deg(X^2) = 2$. Then notice that:
            \[
            p - r = qX^2 \in I \ \implies \ p \sim r
            \]
            Thus, any polynomial is in the equivalence class of its remainder. Since the remainder $r$ has degree 0 or 1, it means that $p$ is in one of $E_I(0), E_I(1), E_I(X), E_I(X + 1)$. Thus:
            \[
            (R/I) = \{E_I(0), E_I(1), E_I(X), E_I(X + 1)\}
            \]
            \item \textbf{Elements in $R/J$} Working in a similar, we will see that:
            \[
            (R/J) = \{E_J(0), E_J(1), E_J(X), E_J(X + 1)\}
            \]
        \end{enumerate}
        \item \textbf{Behaviour of Elements}
        We have reduced the elements of $R/I$ to a set of 4 elements. We now need to see hwo they interact with each other througha  multiplication table:
        \begin{table}[H]
            \small
            \centering
            \begin{tabular}{|c|c|c|c|c|}
            \hline
                       & $E_I(0)$ & $E_I(1)$ & $E_I(X)$ & $E_I(X + 1)$  
                 \\ \hline
               $E_I(0)$ & $E_I(0 \cdot 0) = E_I(0)$ & $E_I(0 \cdot 1) = E_I(0)$ & $E_I(0 \cdot X) = E_I(0)$ & $E_I(0 \cdot (X+1)) = E_I(0)$
                \\\hline
                $E_I(1)$ & $E_I(1 \cdot 0) = E_I(0)$ & $E_I(1 \cdot 1) = E_I(1)$ & $E_I(1 \cdot X) = E_I(X)$ & $E_I(1 \cdot (X+1)) = E_I(X+1)$
                \\\hline
                $E_I(X)$ & $E_I(X \cdot 0) = E_I(0)$ & $E_I(X \cdot 1) = E_I(X)$ & $ E_I(X^2) = E_I(0)$ & $E_I(X^2+X) = E_I(X)$
                \\\hline
                $E_I(X+1)$ & $E_I((X+1) \cdot 0) = E_I(0)$ & $E_I((X+1) \cdot 1) = E_I(X+1)$ & $ E_I(X^2 + X) = E_I(X)$ & $E_I(X^2+2X + 1) = E_I(1)$
                \\\hline
            \end{tabular}
            \caption{Here we use facts like $X^2 \sim 0$ and $X^2 + X \sim X$ to simplify. Also, don't forget that $2 = 0$ in $\F_2$.}
        \end{table}
        \begin{table}[H]
            \small
            \centering
            \begin{tabular}{|c|c|c|c|c|}
            \hline
                       & $E_J(0)$ & $E_J(1)$ & $E_J(X)$ & $E_J(X + 1)$  
                 \\ \hline
               $E_J(0)$ & $E_J(0 \cdot 0) = E_J(0)$ & $E_J(0 \cdot 1) = E_J(0)$ & $E_J(0 \cdot X) = E_J(0)$ & $E_J(0 \cdot (X+1)) = E_J(0)$
                \\\hline
                $E_J(1)$ & $E_J(1 \cdot 0) = E_J(0)$ & $E_J(1 \cdot 1) = E_J(1)$ & $E_J(1 \cdot X) = E_J(X)$ & $E_J(1 \cdot (X+1)) = E_J(X+1)$
                \\\hline
                $E_J(X)$ & $E_J(X \cdot 0) = E_J(0)$ & $E_J(X \cdot 1) = E_J(X)$ & $ E_J(X^2) = E_J(X+1)$ & $E_J(X^2+X) = E_J(1)$
                \\\hline
                $E_J(X+1)$ & $E_J((X+1) \cdot 0) = E_J(0)$ & $E_J((X+1) \cdot 1) = E_J(X+1)$ & $ E_J(X^2 + X) = E_J(1)$ & $E_J(X^2+2X + 1) = E_J(X)$
                \\\hline
            \end{tabular}
            \caption{Here we use facts like $X^2 \sim X + 1$ (since $X^2 - (X + 1) = X^2 - X - 1 = X^2 + X + 1 \in J$ and $X^2 + X \sim 1$ (since $X^2 + X - 1 = X^2 + X + 1 \in J$) to simplify. Also, don't forget that $2 = 0$ in $\F_2$.}
        \end{table}
        Now notice: in $R/J$ every non-zero element has an inverse, so $R/J$ is a \textbf{field} with 4 elements. On the other hand, $R/I$ is \textbf{not}, since it has a \textbf{zero-divisor} (for example $E_I(X)$).
    \end{enumerate}
\end{itemize}

\subsubsection{Exercises (TODO)}

\begin{questions}

\question \textbf{Let $R$ be a ring, and $I$ an ideal of $R$. Show that if $R$ is commutative, then so is $R/I$.}

\question \textbf{Let $R$ be a ring, and $I$ an ideal of $R$. Show that $R/I$ is a non-zero ring if and only if $I \neq R$.}

\question \textbf{Let $R$ be a ring, and $I$ a \textit{proper} ideal of $R$ (so $I \neq R$. Show that if $r \in R^\times$, then $E(r) \in (R/I)^\times$, and $(E(r))^{-1} = E(r^{-1})$.}

\end{questions}

\subsection{Theorem: The Universal Property of Factor Rings}

In the \textbf{Very Important Remark} \eqref{vir} (the \textbf{universal property of the set of equivalence classes}, we showed how there are 2 ways to go between sets $X,Z$: one that is direct ($f : X \to Z$) and one that is indirect ($g : X \to X/\sim \to Z$, with $g = \bar{g} \circ can$, where $\bar{g}$ is a unique mapping $\bar{g}(E(r)) = f(r)$).
This theorem considers the same case, but adapted to \textbf{rings}.

\textbox{\label{utfr}
Let $R$ be a \textbf{ring}, and $I$ an \textbf{ideal} of $R$.
\\
\begin{enumerate}
    \item The \textbf{canonical mapping}:
    \[
    can : R \to (R/I) \qquad can(r) = E(r), \forall r \in R
    \]
    is a \textbf{surjective ring homomorphism}, with \textbf{kernel}:
    \[
    ker(can) = I
    \]
    \item If:
    \[
    f : R \to S
    \]
    is a \textbf{ring homomorphism} and:
    \[
    f(I) = \{0_S\}
    \]
    so that $I \subseteq ker(f)$, then there is a \textbf{unique ring homomorphism}:
    \[
    \bar{f} : (R/I) \to S
    \]
    such that:
    \[
    f = \bar{f} \circ can
    \]
    \pic{homotri.png}{0.5}
\end{enumerate}
[Theorem 3.6.7]
}

\begin{proof}

\begin{enumerate}
    \item \textbf{The Canonical Mapping is a Surjective Ring Homomorphism With Kernel $I$}
    \begin{enumerate}
        \item \textbf{Surjective Mapping}
        This easy to see. Any $E(r)$ is produced by at least one element in $R$. By the pigeonhold principle, every possible $E(r)$ must be mapped to by some element in $R$.
        \item \textbf{The Kernel is $I$} If $i \in I$, then by properties of ideals:    
        \[
        i - 0_R \in I \ \implies \ i \sim 0 \ \implies \ E(i) = E(0) = 0_{R/I}
        \]
        Any other $i \not\in I$ won't have an equivalence relation with $0_R$. Hence, $ker(can) = I$.
        \item \textbf{The Mapping is a Ring Homomorphism} This follows from how \textbf{addition} and \textbf{multiplication} are defined in the \textbf{factor ring} $R/I$:
        \[
        can(x) + can(y) = E(x) + E(y) = E(x + y) = can(x+y)
        \]
        \[
        can(x)can(y) = E(x)E(y) = E(xy) = can(xy)
        \]
    \end{enumerate}
    \item \textbf{Existence of Unique Ring Homomorphism $\bar{f}$}
    \begin{enumerate}
        \item \textbf{Existence of Unique Mapping $\bar{f}$}
        Since $f(I) = \{0_S\}$, then:
        \[
        f(E(x)) = \{f(x + i) \ | \ i \in I\} = \{f(x) + f(i) \ | \ i \in I\} = \{f(x)\}
        \]
        Define:
        \[
        \bar{f}(E(x)) = f(x)
        \]
        such that:
        \[
        f(E(x)) = \{\bar{f}(E(x))\}
        \]
        Then, $\bar{f}$ is the only mapping satisfying $f = \bar{f} \circ can$.
        \item \textbf{$\bar{f}$ is a Ring Homomorphism}
        \[
        \bar{f}(E(x) + E(y)) = \bar{f}(E(x+y)) = f(x+y) = f(x) + f(y) = \bar{f}(E(x)) + \bar{f}(E(y))
        \]
        \[
        \bar{f}(E(x)E(y)) = \bar{f}(E(xy)) = f(xy) = f(x)f(y) = \bar{f}(E(x))\bar{f}(E(y))
        \]
    \end{enumerate}
\end{enumerate}

\end{proof}


\subsection{Theorem: First Isomorphism Theorem for Rings}

\textbox{\label{fitr}
Let $R$ and $S$ be \textbf{rings}.
\\
Then, every \textbf{ring homomorphism}:
\[
f : R \to S
\]
induces a \textbf{ring isomorphism}:
\[
\bar{f} : (R / ker(f)) \to im(f)
\]
This \textbf{isomorphism} is nothing but:
\[
\bar{f}(r + ker(f)) = f(r)
\]
[Theorem 3.6.9]
}

\begin{proof}

Notice, $ker(f)$ is an ideal, so $R / I$ is a ring; similarly, $im(f)$ is a subring, so a \textbf{ring}. Moreover, by definition of the kernel, we must have that $f(ker(f)) = \{0_S\}$. Hence, by the  \textbf{universal property of factor rings}, we have that $\bar{f}$ is a homomorphism.

\bigskip

Clearly, it is also \textbf{surjective} (each $f(r) \in im(f)$ is produced by at least one element in each equivalence class $r + ker(f)$).

\bigskip

Moreover, $ker(\bar{f}) = 0 + ker(f) = ker(f)$ (recall $0 + ker(f)$ is nothing but $E(0)$. If $E(r) = ker(f)$, clearly $\bar{f}(E(r)) = f(r) = 0_S$, by definition of the kernel. No other equivalence class achieves this. Hence, since the kernel only contains the additive identity, the homomorphism must be \textbf{injective}.

\bigskip

Thus, $\bar{f}$ is a bijective homomorphism - an isomorphism.

\end{proof}

\subsubsection{Examples}

\begin{itemize}
    \item if $R = \mathbb{R}[X]$ and $I = {}_R\langle X^2 + 1 \rangle$ (the \textbf{ideal} generated by $X^2 + 1$, or in other words, the set of all polynomials with $X^2 + 1$ as a factor), then $R/I$ is not only a \textbf{ring}, but it is \textbf{isomorphic} to the \textbf{complex numbers}
    \begin{itemize}
        \item we can \textbf{factorise} polynomials uniquely ($P = AQ + B$, with $P,Q \in R$, $deg(B) < deg(Q)$)
        \item in particular, we can write $P \in R$ as:
        \[
        P = A(X^2 + 1) + B
        \]
        \item since $Q = X^2 + 1$, and $deg(B) < deg(Q)$ we must have:
        \[
        B = a + bX, \qquad a,b \in \mathbb{R} 
        \]
        \item now consider the \textbf{evaluation homomorphism}:
        \[
        f : \mathbb{R}[X] \to \mathbb{C}
        \]
        defined by evaluation $P \in \mathbb{R}[X]$ at $\sqrt{-1}$
        \item clearly, $f(P) = f(B)$, since $\sqrt{-1}$ is a root of $X^2 + 1$, so:
        \[
        f(P) = f(B) = a + b\sqrt{-1}
        \]
        \item clearly, $f$ is surjective
        \item moreover, $P \in ker(f)$ \textbf{if and only if} $a = b = 0$, which in particular means that:
        \[
        ker(f) = {}_\mathbb{R}[X]\langle X^2 + 1 \rangle
        \]
        \item by the \textbf{first isomorphism theorem for rings}, we thus have an \textbf{isomorphism}:
        \[
        \bar{f} : (\mathbb{R}[X] / {}_{\mathbb{R}[X]}\langle X^2 + 1 \rangle) \to \mathbb{C}
        \]
    \end{itemize}
\end{itemize}

\section{Modules}

Just as \textbf{rings} are the generalisation of fields, we introduct \textbf{modules} as the generalisation of\textbf{ vector spaces}.

\subsection{Defining Modules}

\begin{itemize}
    \item \textbf{What is a left module?}
    \begin{itemize}
        \item a \textbf{left module} is defined over \textbf{rings}
        \item it consists of an \textbf{abelian group}:
        \[
        M = (M,\dot{+})
        \]
        armed with a mapping:
        \[
        R \times M \to M
        \]
        \[
        (r,a) \to r\cdot a
        \]
        \item \textbf{left modules} must satisfy:
        \[
        r(a \dot{+} b) = (ra) \dot{+} (rb)
        \]
        \[
        (r + s)a = (ra) \dot{+} (sa)
        \]
        \[
        r(sa) = (rs)a
        \]
        \[
        1_Ra = a
        \]
    \end{itemize}
    \item \textbf{What is an R-Module?}
    \begin{itemize}
        \item a module defined over the \textbf{ring} $R$
    \end{itemize}
    \item \textbf{How do right modules differ from left modules?}
    \begin{itemize}
        \item a module in which multiplication by rings is defined via:
        \[
        (r,a) \to a \cdot r
        \]
    \end{itemize}
    \item \textbf{What is the trivial module?}
    \begin{itemize}
        \item the singleton $\{0\}$ for any ring $R$
    \end{itemize}
    \item \textbf{What is a direct sum?}
    \begin{itemize}
        \item given $R$-modules:
        \[
        M_1, M_2, \ldots, M_n
        \]
        their \textbf{cartesian product}:
        \[
        M_1 \times M_2 \times\ldots \times M_n
        \]
        alongside:
        \[
        (a_1, \ldots, a_n) + (b_1, \ldots, b_n) = (a_1 + b_1, \ldots, a_n + b_n)
        \]
        \[
        r(a_1, \ldots, a_n) = (ra_1, \ldots, ra_n)
        \]
        is an $R$-module
        \item denoted:
        \[
        M_1 \oplus M_2 \oplus \ldots \oplus M_n
        \]
        is the \textbf{direct sum}
    \end{itemize}
    \item \textbf{How do R-Modules and F-Vector Spaces differ?}
    \begin{itemize}
        \item since modules are defined over \textbf{rings}, multiplication by $R$ might \textbf{not} have inverses defined
        \item this means that if for example:
        \[
        rm = 0
        \]
        we can't assume that $r = 0$ or $m = 0$. For example, if:
        \[
        R = \mathbb{Z}, \qquad (M,+) = (\mathbb{Z}_4, +)
        \]
        then:
        \[
        2 \cdot \bar{2} = \bar{4} = \bar{0}
        \]
        \item amongst other things, this means that the notion of \textbf{linear independence} no longer makes sense in \textbf{modules}, since \textbf{linear combinations} can be 0, with not all ring scalars being 0
    \end{itemize}
\end{itemize}

\subsubsection{Examples}

\begin{itemize}
    \item $F$-vector spaces are just $R$-modules in which the \textbf{ring} $R$ is a field $F$
    \item $\mathbb{Z}$-modules are \textbf{abelian groups}. Indeed, any abelian group $M$ is a $\mathbb{Z}$-module.
    \item if $I$ is an ideal of a ring $R$, then $I$ is an $R$-module, under multiplication in the ring. In fact, $R$ is an $R$-module.
    \begin{itemize}
        \item for example, $\mathbb{Z}$ and $\mathbb{Z}_6$ are both modules
        \item ideals exploit the fact that if an element of $r \in R$ multiplies $i \in R$, then $ir, ri \in I$
    \end{itemize}
\end{itemize}

\subsubsection{Exercises (TODO)}

\begin{questions}

\question \textbf{Let $S$ be a ring, and let $R = Mat(n;S)$. Let $M = S^n$. Show that $M$ is an $R$-module under the operations of componentwise addition and amtrix multiplication.}

\question \textbf{Let $V$ be an $F$-vector space for some field $F$ and let $\phi \in End(V)$ be an endomorphism of $V$. Show that $V$ is an $F[X]$-module under the operation:
\[
\left(\sum_{i = 0}^m a_iXî\right)\vec{v} = \sum_{i = 0}^m a_i\phi^{m}(\vec{v})
\]
For better understanding, we are multiplying a vector $\vec{v}$ by a polynomial with coefficients in a field. This multiplication then needs to result in a vectors in the vector space. We denote the $F[X]$-module via $V_\phi$.}

As an example for this, consider:
\[
R = \mathbb{C}[X] \qquad (M,+) = \mathbb{C}^n
\]
We can define the endomorphism $\phi$ as:
\[
\phi(\vec{v}) = A\vec{v} \qquad \vec{v} \in m, A \in Mat(n;\mathbb{C})
\]
and then define ring multiplication as:
\[
p(X)\vec{v} := p(A)\vec{v}, \qquad p(X) \in R 
\]
As a concrete example, define:
\[
A = \begin{pmatrix}
0 & 1 \\
0 & 0
\end{pmatrix}
\]
and
\[
q(X) = X^2 + 2X + 3
\]
Then:
\[
q(A) = A^2 + 2A + 3I = \begin{pmatrix}
3 & 2 \\
0 & 3
\end{pmatrix}
\]
Such that:
\[
q(X) \cdot (0 \ 1)^T = \begin{pmatrix}
3 & 2 \\
0 & 3
\end{pmatrix}
\begin{pmatrix}
0 \\
1
\end{pmatrix}
=
\begin{pmatrix}
2 \\
3
\end{pmatrix}
\]

\end{questions}

\subsection{Lemma: Module Hygiene}

\textbox{
Let $R$ be a ring, and $M$ an $R$-module.
\begin{enumerate}
    \item $0_R \cdot a = 0_M, \qquad \forall a \in M$
    \item $r0_M = 0_M, \qquad \forall r \in R$
    \item $(-r)a = r(-a) = -(ra), \qquad \forall r \in R, a \in M$
\end{enumerate}
}

\subsection{Module Homomorphisms}

\begin{itemize}
    \item \textbf{What is a module homomorphism?}
    \begin{itemize}
        \item let $R$ be a ring, and let $M,N$ be $R$-modules
        \item a $R$-\textbf{homomorphism} is a mapping:
        \[
        f : M \to N
        \]
        satisfying:
        \[
        f(a+b) = f(a) + f(b)
        \]
        \[
        f(ra) = rf(a)
        \]
    \end{itemize}
    \item \textbf{What results from composing module homomorphisms?}
    \begin{itemize}
        \item you obtain another \textbf{homomorphism}
    \end{itemize}
    \item \textbf{What is an R-Module Isomorphism?}
    \begin{itemize}
        \item a \textbf{bijective} homomorphism
    \end{itemize}
    \item \textbf{What is the kernel of an $R$-homomorphism?}
    \begin{itemize}
        \item the set:
        \[
        ker(f) = \{a \in M \ | \ f(a) = 0_N\} \subseteq M
        \]
    \end{itemize}
    \item \textbf{What is the image of an $R$-homomorphism?}
    \begin{itemize}
        \item the set:
        \[
        im(f) = \{f(a) \ | \ a \in M\} \subseteq N
        \]
    \end{itemize}
\end{itemize}

\subsubsection{Examples}

\begin{itemize}
    \item the mapping $f(a) = 0_N$ is \textbf{always} an $R$-homomorphism
    \item if $R$ is a field, module homomorphism are the standard \textbf{linear mappings}
    \item any \textbf{group} homomorphism between abelian groups is also a  $\mathbb{Z}$-homomorphism
    \item consider $R = \mathbb{C}[X]$, and consider 2 modules:
    \[
    M = \mathbb{C}_A^2 \qquad N = \mathbb{C}_B^2
    \]
    where:
    \[
    A = \begin{pmatrix}
    0 & 1 \\
    0 & 0
    \end{pmatrix}
    B = \begin{pmatrix}
    2 & -1 \\
    4 & -2
    \end{pmatrix}
    \]
    We can then define a homomorphism:
    \[
    f : \mathbb{C}_A^2 \to \mathbb{C}_B^2
    \]
    defined by:
    \[
    \vec{v} \to T\vec{v}, \qquad \vec{v} \in M, T \in Mat(2;\mathbb{C})
    \]
    Recall, multiplication in the module is defined by replacing each $X$ in the polynomial in $\mathbb{C}[X]$ by the given matrix (A or B). Hence, the homomorphism is defined by:
    \[
    f(X\vec{v}) = f(A\vec{v}) = T(A\vec{v})
    \]
    But notice, this must be an element in $N$. By properties of homomorphisms:
    \[
    f(X \vec{v}) = Xf(\vec{v}) = B(T\vec{v})
    \]
    Hence, if $T$ exists, it must satisfy:
    \[
    TA = BT
    \]
    Let:
    \[
    T = \begin{pmatrix}
    a & b \\
    c & d
    \end{pmatrix}
    \]
    Then:
    \[
    AT = 
    \begin{pmatrix}
    0 & a \\
    0 & c
    \end{pmatrix}
    \]
    \[
    TB = 
    \begin{pmatrix}
    2a-c & 2b-d \\
    4a - 2c & 4b-2d
    \end{pmatrix}
    \]
    Hence, we have 4 variables, and 4 sets of linear equations:
    \[
    2a - c = 0 \ \implies \ c = 2a
    \]
    \[
    4a - 2c = 0 \ \implies \ c = 2a
    \]
    \[
    4b - 2d = c
    \]
    \[
    2b - d = a
    \]
\end{itemize}
Notice, the last 2 equations coincide with the fact that $c = 2a$. Overall, this system has infinitely many solutions, such that:
\[
T = \begin{pmatrix}
2b-d & b \\
4b - 2d & d
\end{pmatrix}
\]

\subsubsection{Exercises (TODO)}

\begin{questions}

\question \textbf{Let $F$ be a field, and let $V = F^2$ and $W = F^3$ be $F$-vector spaces. Define:
\[
\phi = \begin{pmatrix}
0 & 1 \\
0 & 0
\end{pmatrix}
\qquad 
\psi = \begin{pmatrix}
0 & 1 & 0 \\
0 & 0 & 1 \\
0 & 0 & 0
\end{pmatrix}
\]
and consider the $F[X]$-modules $V_\phi$, $W_\psi$. Show that:
\[
f : V_\phi \to W_\psi, \qquad \begin{pmatrix}
x \\
y
\end{pmatrix}
\to 
\begin{pmatrix}
x \\
y \\
0
\end{pmatrix}
\]
\[
f : W_\psi \to V_\phi, \qquad \begin{pmatrix}
x \\
y \\
x
\end{pmatrix}
\to 
\begin{pmatrix}
z \\
0 
\end{pmatrix}
\]
are $F[X]$-homomorphisms}

\end{questions}

\subsection{Submodules}

\begin{itemize}
    \item \textbf{What are submodules?}
    \begin{itemize}
        \item non-empty \textbf{subsets} of an $R$-module, which are themselves $R$-modules, with respect to the operations in the $R$-module
    \end{itemize}
\end{itemize}

\subsubsection{Examples}

\begin{itemize}
    \item a \textbf{submodule} of an $F$-vector space is a subspace
    \item the \textbf{submodules} of a $\mathbb{Z}$-module are the subgroups of its corresponding group
    \item the \textbf{submodules} of a ring are its ideals
    \item consider a field $F$ and the $F[X]$-module $W_\psi$ defined using the matrix:
    \[
    \psi = \begin{pmatrix}
    0 & 1 & 0 \\
    0 & 0 & 1 \\
    0 & 0 & 0
    \end{pmatrix}
    \]
    (recall $W_\psi$ is an $F[X]$-module, where its elements are in $F^3$, and multiplication by a matrix $F[X]$ is defined as multiplication of $\vec{v} \in F^3$ by $F[\psi]$). Then the subspaces:
    \begin{itemize}
        \item $\langle \vec{e}_1 \rangle = \{0\}$
        \item $\langle \vec{e}_1, \vec{e}_2 \rangle = \{0\} \cup \{k\vec{e}_1 \ | \ k \in F\}$ (since $\phi \vec{e}_2 = \vec{e}_1$, and $\phi \vec{e}_1 = 0$)
    \end{itemize}
    are $F[X]$-submodules of $W_\psi$, but $\langle \vec{e}_2 \rangle$ (since $\phi \vec{e}_2 = \vec{e}_1$, but $\vec{e}_1$ is not part of the generating set).
    \item again, consider:
    \[
    R = \mathbb{C}[X] \qquad M = \mathbb{C}_A^2
    \]
    where multiplication by polynomials in $M$ is defined by multiplying $\vec{v} \in M$ by:
    \[
    A = \begin{pmatrix}
    0 & 1 \\
    0 & 0
    \end{pmatrix}
    \]
    The question to consider is: the 1-d subspace of $\mathbb{C}^2$ given by:
    \[
    L = \{\lambda (x,y) \ | \ \lambda \in \mathbb{C}\}
    \]
    is definitely closed under addition and scalar multiplication; is it closed under multiplication by polynomials? That is, is it a submodule of $M$?
    \bigskip
    Consider $p(X) = \sum_{i = 0}^n p_iXî \in \mathbb{C}[X]$. Moreover, notice that:
    \[
    A^2 = Mat(0) \ \implies \ A^k = Mat(0), \qquad \forall k \in [2,n]
    \]
    Thus:
    \[
    p(A) = p_0I_2 + p_1A = \begin{pmatrix}
    p_0 & p_1 \\
    0 & p_0
    \end{pmatrix}
    \]
    Hence, we ask whether:
    \[
    p(A)(x,y) = \begin{pmatrix}
    p_0 & p_1 \\
    0 & p_0
    \end{pmatrix}
    \begin{pmatrix}
    x \\
    y
    \end{pmatrix}
    = 
    \begin{pmatrix}
    p_0x + p_1y \\
    p_0y
    \end{pmatrix}
    \in L
    \]
    We thus need to find suitable $x,y$, such that:
    \[
    \begin{pmatrix}
    p_0x + p_1y \\
    p_0y
    \end{pmatrix}
    =
    \begin{pmatrix}
    \lambda x \\
    \lambda y
    \end{pmatrix}
    \]
    Since the second entry only depends on $y$, we focus on that first. There are 2 cases to consider:
    \begin{enumerate}
        \item $\boldsymbol{y \neq 0}$
        In this case, and since $\mathbb{C}$ is an integral domain, we must have that $p_0 = \lambda$. Thus, in the first entry:
        \[
        p_0x + p_1y = \lambda x \ \implies \ \lambda x + p_1y = \lambda x \ \implies \ p_1y = 0
        \]
        Now, since $y \neq 0$, this is only possible if $p_1 = 0$. But we need to consider every possible polynomial in $\mathbb{C}[X]$, so this is not possible. Hence, the only alternative is that $y = 0$.
        \item $\boldsymbol{y = 0}$
        In this case, $p_0$ can be anything in $\mathbb{C}$. Then:
        \[
        p_0x + p_1y = \lambda x \ \implies \ p_0x = \lambda x \ \implies \ p_0 = \lambda
        \]
    \end{enumerate}
    Thus, for any $x$, and for $y = 0$, $L$ defines a submodule.
\end{itemize}

\subsection{Proposition: Test for a Submodule}

\textbox{
Let $R$ be a \textbf{ring}, and $M$ a module over $R$. 
\\
A subset $M' \subseteq M$ is a \textbf{submodule} if and only if:
\begin{enumerate}
    \item $0_M \in M'$
    \item $a,b \in M' \ \implies \ a - b \in M'$
    \item $r \in R, a \in M' \ \implies \ ra \in M'$
\end{enumerate}
[Proposition 3.7.20]
}

\begin{proof}

If $M'$ is a submodule, these properties hold (since they are properties of modules). Alternatively, assume that $M'$ satisfies the conditions. Then, recall the test of a (finite) subgroup. If $G$ is a group, $H$ is a subgroup if and only if:
\begin{itemize}
    \item $H \neq \emptyset$
    \item $h,k \in H \ \implies \ hk^{-1} \in H$
\end{itemize}
Condition (1) means that $M$ is not empty, and condition (2) ensures that $a - b \in M'$. Hence, $M'$ is a subgroup of $M$. By (3), we know that we have:
\[
R \times M' \to M'
\]
The remaining properties of a module are satisfied by the fact that $M'$ is a subset of $M$. Hence, $M'$ must be a submodule.

\end{proof}

\subsection{Lemma: Kernel and Image as Submodules}

\textbox{
Let:
\[
f : M \to N
\]
be a \textbf{module homomorphism}. Then:
\begin{itemize}
    \item $ker(f)$ is a \textbf{submodule} of $M$
    \item $im(f)$ is a \textbf{submodule} of $N$
\end{itemize}
[Lemma 3.7.21]
}

\begin{proof}

\begin{enumerate}
    \item \textbf{The Kernel is a Submodule}
    \begin{itemize}
        \item since $f(0_M) = 0_N$, $0_M \in ker(f)$ 
        \item if $a,b \in ker(f)$ then:
        \[
        f(a) - f(b) = 0_M \ \implies \ f(a - b) = 0 \ \implies \ a-b \in ker(f)
        \]
        \item if $r \in R, a \in ker(f)$:
        \[
        rf(a) = r0_M = 0_M \ \implies \ f(ra) = 0_M \ \implies \ ra \in ker(f)
        \]
    \end{itemize}
    Hence, by the test for a submodule, the kernel is a submodule.
    \item \textbf{The Image is a Submodule}
    \begin{itemize}
        \item since $f(0_M) = 0_N$, $0_N \in im(f)$ 
        \item if $a,b \in im(f)$, then $\exists a',b' \in M$ such that:
        \[
        f(a') = a \qquad f(b') = b
        \]
        But then, by properties of the homomorphism:
        \[
        f(a' - b') = a-b
        \]
        Since $a'-b' \in M$ (by definition of a module), it follows that $a - b \in N$
        \item if $r \in R, a \in im(f)$, then $\exists a' \in M$ such that:
        \[
        f(a') = a
        \]
        But then:
        \[
        rf(a') = ra \ \implies \ f(ra') = ra
        \]
        so $ra \in im(f)$
    \end{itemize}
    Hence, by the test for a submodule, the image is a submodule.
\end{enumerate}

\end{proof}



\subsection{Lemma: Injectivity and Kernel}

\textbox{
Let $R$ be a \textbf{ring}, with $M,N$ as \textbf{$R$-modules}.
\\
Let:
\[
f : M \to N
\]
be a \textbf{module homomorphism}. Then, $f$ is injective \textbf{if and only if}:
\[
ker(f) = \{0_M\}
\]
[Lemma 3.7.22]
}

\begin{proof}
This follows directly from the fact that this property is true for group homomorphisms.
\end{proof}

\subsection{Generating Submodules}

\begin{itemize}
    \item \textbf{What is a generated module?}
    \begin{itemize}
        \item consider a ring $R$, with $R$-module $M$, and a subset $T \subseteq M$
        \item the \textbf{submodule of $M$ generated by $T$} is the submodule:
        \[
        {}_R\langle T\rangle = \left\{\sum_{i = 1}^m r_it_i \ | \ r_i \in R, t_i \in T\right\}
        \]
        \item if $T = \emptyset$, then ${}_R\langle T\rangle$ contains $0_M$
    \end{itemize}
    \item \textbf{What is a finitely generated module?}
    \begin{itemize}
        \item a \textbf{module} generated by a \textbf{finite set}:
        \[
        M = {}_R\langle T\rangle
        \]
    \end{itemize}
    \item \textbf{What is a cyclic module?}
    \begin{itemize}
        \item a \textbf{module} generated by a \textbf{single} element:
        \[
        M = {}_R\langle t\rangle, \qquad t \in M
        \]
    \end{itemize}
\end{itemize}

\subsubsection{Examples}

\begin{itemize}
    \item a \textbf{cyclic} $\mathbb{Z}$-module is equivalent to a \textbf{cyclic abelian group}
    \item the \textbf{ideal} generated by a subset $T$ of a \textbf{commutative ring} $R$ is equivalent to a \textbf{submodule} of $R$ generated by $T$
    \item a \textbf{principal ideal} of $R$ is equivalent to a \textbf{cyclic submodule} of $R$
    \item if $F$ is a field, and $W_\psi$ is defined as above, then $w_\psi$ is a \textbf{cyclic} $F[X]$-module generated by $\vec{e}_3$
    \item $0_M$ is \textbf{generates} a \textbf{cyclic submodule} $\{0_M\}$ of \textbf{any} module
\end{itemize}

\subsection{Lemma: Smallest Submodule Containing a Subset}

\textbox{
If $T \subseteq M$, then:
\[
{}_R\langle T \rangle 
\]
is the \textbf{smallest submodule} of $M$ containing $T$.
[Lemma 3.7.28]
}

\subsection{Lemma: Intersection of Submodules}

\textbox{
The \textbf{intersection} of any \textbf{collection of modules} is a \textbf{module}.
[Lemma 3.7.29]
}

\subsection{Lemma: Addition of Submodules}

\textbox{
If $M_1,M_2$ are \textbf{submodules} of $M$, then:
\[
M_1 + M_2 := \{m_1 + m_2 \ | \ m_1 \in M_1, m_2 \in M_2\}
\]
is also a \textbf{submodule} of $M$.
[Lemma 3.7.30]
}

\subsection{Theorem: Factor Modules}

\begin{itemize}
    \item \textbf{What are cosets in modules?}
    \begin{itemize}
        \item let:\begin{itemize}
            \item $R$ be a \textbf{ring}
            \item $M$ be a \textbf{module}
            \item $N$ a \textbf{submodule} of $M$
        \end{itemize} 
        \item similarly to before, we can define an \textbf{equivalence relation}:
        \[
        a \sim b \ \iff \ a - b \in N, \qquad a,b \in M
        \]
        \item for $a \in M$, the \textbf{equivalence class} of this relation is the \textbf{coset of $a$ with respect to $N$ in $M$}:
        \[
        E(a) = a + N = \{a + n \ | n \in N\}
        \]
    \end{itemize}
    \item \textbf{How are factor modules defined?}
    \begin{itemize}
        \item the \textbf{factor of $M$ by $N$} (or the \textbf{quotient of $M$ by $N$} is the set:
        \[
        M/N = M/\sim
        \]
        of \textbf{all cosets/equivalence classes} of $N$ in $M$.
    \end{itemize}
\end{itemize}

\subsubsection{Examples}

Let $R = \mathbb{R}, M = \mathbb{R}^4$ and:
\[
N = \{(x_1, x_2, x_3, x_4) \ | \ x_1 = 2x_3, x_2 = 4x_4\}
\]
What is $M/N$?
\bigskip
We know its an $R$ module over a field, so $M/N$ is a vector space. First, lets consider what the bases of $M,N$ are. For $N$ its simple:
\[
\{(2,0,1,0), (0,4,0,1)\}
\]
For $M$, we can extend the basis for $N$:
\[
\{(2,0,1,0), (0,4,0,1), (1,0,0,0), (0,1,0,0)\}
\]
Indeed, this is a basis, since the vectors are linearly independent, and:
\[
(x_1, x_2, x_3, x_4) = x_3(2,0,1,0) + x_4(0,4,0,1) + (x_1 - 2x_3)(1,0,0,0) + (x_2 - 4x_4)(0,1,0,0)
\]
Now, recall that:
\[
a \sim b \ \iff \ a - b \in N
\]
We claim that a basis for $M/N$ is given by:
\[
\{(1,0,0,0) + N, (0,1,0,0) + N\}
\]
For this we need 2 things:

\begin{enumerate}
    \item \textbf{Generation}
    Take any element in $M/N$:
    \[
    (x_1,x_2,x_3,x_4) + N
    \]
    Then, it is clear that:
    \[
    ((x_1, x_2, x_3, x_4) + N) - (((x_1 - 2x_3)(1,0,0,0) + N) + ((x_2 - 4x_4)(0,1,0,0) + N)) = (x_3(2,0,1,0) + N) + (x_4(0,4,0,1) + N)
    \]
    But notice, $\{(2,0,1,0), (0,4,0,1)\}$ is a basis for $N$, so:
    \[
    ((x_1, x_2, x_3, x_4) + N) - (((x_1 - 2x_3)(1,0,0,0) + N) + ((x_2 - 4x_4)(0,1,0,0) + N)) \in N
    \]
    or in other words, $(x_1, x_2, x_3, x_4) + N$ and $((x_1 - 2x_3)(1,0,0,0) + N) + ((x_2 - 4x_4)(0,1,0,0) + N$ are equivalent in the cosets, so:
    \[
    (x_1, x_2, x_3, x_4) + N = ((x_1 - 2x_3)(1,0,0,0) + N) + ((x_2 - 4x_4)(0,1,0,0) + N)
    \]
    Hence, any element in $M/N$ is generated by our claimed basis.
    \item \textbf{Linear Independence}
    It is clear that if:
    \[
    (\alpha (1,0,0,0) + \beta (0,1,0,0)) + N = (0,0,0,0) + N
    \]
    we can only have $\alpha = \beta = 0$, so the generating set is lienarly independent.
\end{enumerate}

\subsection{Theorem: Factor Module Operations}

\textbox{
Let $R$ be a ring, and let $M,N$ be $R$-moduls. For $a,b \in M$ and $r \in R$.
\\
For the \textbf{factor module} $M/N$, define \textbf{addition} via:
        \[
        (a + N) + (b + N) = (a+b) + N \qquad E(a) + E(b) = E(a+b)
        \]
 and \textbf{multiplication} via:
        \[
        r(a + N) = (ra) + N
        \]
[Theorem 3.7.31]
}

\begin{proof}

As before, we need to show that not only $M/N$ is a module, but also that \textbf{addition} and \textbf{multiplication} are \textbf{well-defined}.

\bigskip

Addition is well-defined, since additively, modules are abelian groups, so the proof for \textbf{factor rings} applies.

\bigskip

For multiplication, consider $a,b \in M$, such that:
\[
E(a) = E(b)
\]
We need to show that:
\[
rE(a) = rE(b)
\]
By properties of modules, this means that $a \sim b \ \implies \ a-b \in N$. Again by properties of modules, $r(a-b) \in N \ \implies \ ra - rb \in N$. Hence:
\[
E(ra) = E(rb) \ \iff \ rE(a) = rE(b)
\]
Thus, multiplication is \textbf{well-defined}.

\bigskip

Lastly, we check that addition defines a group:
\begin{itemize}
    \item $E(0) + E(a) = E(0 + a) = E(a) = E(a + 0) = E(a) + E(0)$
    \item $E(a) + E(-a) = E(a - a) = E(0)$
\end{itemize}

Hence, $M/N$ is indeed a module.

\end{proof}

\subsection{Theorem: The Universal Property of Factor Modules}

\textbox{
Let $R$ be a \textbf{ring}, with $L$ and $M$ as $R$-modules. Let $N$ be a \textbf{submodule} of $M$.
\\
Then:
\begin{enumerate}
    \item The \textbf{canonical mapping}:
    \[
    can : M \to M/N
    \]
    \[
    can(a) = E(a) = a + N, \qquad \forall a \in M
    \]
    is a \textbf{surjective $R$-module homomorphism}, with:
    \[
    ker(can) = n
    \]
    \item If:
    \[
    f : M \to L
    \]
    is an \textbf{$R$-homomorphism} with:
    \[
    f(N) = \{0_L\}
    \]
    (so then $N \subseteq ker(f)$), then there is a \textbf{unique homomorphism}:
    \[
    \bar{f} : M/N \to L
    \]
    \[
    \bar{f}(E(a)) = f(a), \qquad \forall a \in M
    \]
    such that:
    \[
    f = \bar{f} \circ can
    \]
    \pic{modtri.png}{0.5}
\end{enumerate}
[Theorem 3.7.32]
}

\begin{proof}

The proof is completely analogous to the proof of the \textbf{universal property of factor rings} \eqref{utfr}

\end{proof}

\subsection{Theorem: First Isomorphism Theorem for Modules}

\textbox{
Let $R$ be a \textbf{ring}, and let $M,N$ be \textbf{$R$-modules}
\\
Then, every \textbf{$R$-homomorphism}:
\[
f : M \to N
\]
induces an \textbf{$R$-isomorphism}:
\[
\bar{f} : (M/ker(f)) \to im(f)
\]
[Theorem 3.7.33]
}

\begin{proof}
Again, completeley analogous to that of the \textbf{first isomorphism theorem for rings} \eqref{fitr}
\end{proof}

\subsection{Remark: First Isomorphism Theorem for Vector Spaces}

\textbox{
If we pick $R = F$ to be a field, then the above gives us the \textbf{First Isomorphism Theorem for Vector Spaces}.
\\
Similarly to before, we can show that:
\[
dim(M/ker(f)) = dim(M) - dim(ker(f))
\]
Moreover, due to the isomorphism $\bar{f} : (M/ker(f)) \to im(f)$ we know that:
\[
dim(M/ker(f)) = dim(im(f))
\]
which gives us another proof of the \textbf{rank-nullity theorem}.
[Remark 3.7.34]
}

\subsection{Remark: First Isomorphism Theorem for Abelian Groups}

\textbox{
If we pick $R = \mathbb{Z}$, then the above gives us the \textbf{First Isomorphism Theorem for Abelian Groups}, a special case of the \textbf{First Isomorphism Theorem for Groups}.
[Remark 3.7.35]
}

\begin{questions}

\question \textbf{Let $N, K$ be submodules of an $R$–module $M$. Show that $K$ is a submodule of $N + K = \{b + c \ | \  b \in N, c \in K\}$ and $N \cap K$ is a submodule of $N$. Show further that:
\[
\frac{N+K}{N} \cong \frac{N}{N \cap K}
\]
This is the \textit{Second Isomorphism Theorem for Modules}}

\question \textbf{Let $N, K$ be submodules of an $R$–module $M$, where $K \subseteq N$. Show that $N/K$ is a submodule of $M/K$, and that: 
\[
\frac{M/K}{N/K} \cong M/N
\]
This is the \textit{Third Isomorphism Theorem for Modules}}
 
\end{questions}


\end{document}