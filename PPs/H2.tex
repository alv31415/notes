\documentclass{exam}
\usepackage[utf8]{inputenc}

\usepackage[bottom = 2cm, left = 1.5cm, right = 1.5cm]{geometry}

\usepackage{mynotes}

\title{Honours Algebra - Practicing Proof 1}
\author{s1908368}
\date{February 2022}

\begin{document}

\maketitle

\renewcommand{\i}{\sqrt{-1}}

Consider the ring of \textbf{Gaussian Integers}
\[
\mathbb{Z}[\i ] = \{r + \i s \ | \ r,s \in \mathbb{Z}\}
\]
which is a subset of $\mathbb{C}$ and is a \textbf{commutative ring}. We consider $n \in \mathbb{Z}, n \neq 0$ and set $q = n^2 + 1$. We consider the ideal $I_n = {}_{\mathbb{Z}[\i]}\langle \i - n \rangle$. Define $[r]$ to denote the equivalence class of $r \in \mathbb{Z}$.

\begin{questions}

\question \textbf{Prove that $\mathbb{Z}[\i]$ is an integral domain.}

$\mathbb{C}$ is a field, and any field is an integral domain. Hence, if $x,y \in \mathbb{C}$ and $xy = 0$, then either $x = 0$ or $y = 0$ (or both). Since $\mathbb{Z}[\i] \subset \mathbb{C}$, then if there are $x',y' \in \mathbb{Z}[\i]$ with $x'y' = 0$, it must be the case that $x' = 0$ or $y' = 0$, so in particular, $\mathbb{Z}[\i]$ is an integral domain.

\question \textbf{Find the order of $\mathbb{Z}[\i]^\times$, the group of units of $\mathbb{Z}[\i]$, and determine whether or not it is cyclic.}

Let $x \in \mathbb{Z}[\i]$ be a unit of $\mathbb{Z}[\i]$. Then, $\exists y \in \mathbb{Z}[\i]$ such that $xy = 1$. Let:
\[
x = a + \i b \qquad y = c + \i d \qquad a,b,c,d \in \mathbb{Z}
\]
We must have (from distributivity and associativity of rings):
\[
(a + \i b)(c + \i d) = 1 \ \implies \ (ac - bd) + \i(ad + bc) = 1
\]
For this equality to hold, we require:
\[
ac - bd = 1 \qquad ad + bc = 0
\]
Furthermore:
\begin{align*}
    &(a + \i b)(c + \i d) = 1 \\
    \implies & (a + \i b)(c + \i d)(a - \i b)(c - \i d) = (a - \i b)(c - \i d) \\
    \implies & (a + \i b)(a - \i b)(c + \i d)(c - \i d) = (ac - bd) - \i(ad + bc) \\
    \implies & (a^2 + b^2)(c^2 + d^2) = 1 \\
\end{align*}
where we have used the commutativity of $\mathbb{Z}[\i]$, alongside the restrictions $ac - bd = 1, ad + bc = 0$.

\smallskip

Now notice that $a^2 + b^2 \in \mathbb{Z}, c^2 + d^2 \in \mathbb{Z}$, so in particular, we must have that $a^2 + b^2$ and $c^2 + d^2$ are units of $\mathbb{Z}$. The only units of $\mathbb{Z}$ are $1,-1$, so it follows that:
\[
a^2 + b^2 = c^2 + d^2 = 1 \quad \text{\textit{or}} \quad a^2 + b^2 = c^2 + d^2 = -1
\]
The latter case $a^2 + b^2 = c^2 + d^2 = -1$ is impossible, since $a^2 \geq 0, b^2 \geq 0, c^2 \geq 0, d^2 \geq 0$. Hence, we focus only on the case $a^2 + b^2 = 1$. Again, since $a^2 \geq 0, b^2 \geq 0$ this is only possible if $a^2 = 0$ (in which case $a = 0$ and $b = 1$ or $b = -1$) or if $b^2 = 0$ (in which case $b = 0$ and $a = 1$ or $a = -1$). Thus, we get that the only possible units of $\mathbb{Z}[\i]$ are:
\[
\mathbb{Z}[\i]^\times = \{1 + 0\i, -1 + 0\i, 0 + 1\i, 0 - 1\i\} = \{1,-1,\i, -\i\}
\]
(we don't need to check the case of $c^2 + d^2$, since we are focusing on $x = a + \i b$ as the unit)
\smallskip

From the above, it follows that the order of $\mathbb{Z}[\i]^\times$ is 4. Moreover, $\mathbb{Z}[\i]^\times$ is cyclic, since $\i$ generates the group (under multiplication):
\[
\i = \i \qquad (\i)(\i) = -1 \qquad (\i)(\i)(\i) = -\i \qquad (\i)(\i)(\i)(\i) = 1
\]

\question \textbf{Show that the map}
\[
\phi : \mathbb{Z}[\i] \to \mathbb{Z}_q \qquad \phi(r + \i s) = [r] + [n][s], \qquad \forall r,s \in \mathbb{Z}
\]
\textbf{is a surjective ring homomorphism.}

Let $r_1,r_2,s_1,s_2 \in \mathbb{Z}$. Then from Example 3.1.4 we know that:
\[
[r_1 + r_2] = [r_1] + [r_2] \qquad [r_1s_1] = [r_1][s_1]
\]
We check the properties of the homomorphism on $\phi$.
\begin{align*}
    \phi((r_1 + \i s_1) + (r_2 + \i s_2)) &= \phi((r_1 + r_2) + \i (s_1 + s_2)) \\
                                          &= [r_1 + r_2] + [n][s_1 + s_2] \\
                                          &= [r_1] + [r_2] + [n]([s_1] + [s_2]) \\
                                          &= ([r_1] + [n][s_1]) + ([r_2] + [n][s_2]) \\
                                          &= \phi(r_1 + \i s_1) + \phi(r_2 + \i s_2) \\
\end{align*}
Hence $\phi((r_1 + \i s_1) + (r_2 + \i s_2)) = \phi(r_1 + \i s_1) + \phi(r_2 + \i s_2)$.
\begin{align*}
    \phi((r_1 + \i s_1)(r_2 + \i s_2)) &= \phi((r_1r_2 - s_1s_2) + \i (r_1s_2 + r_2s_1)) \\
                                          &= [r_1r_2 - s_1s_2] + [n][r_1s_2 + r_2s_1] \\
                                          &= [r_1r_2] - [s_1s_2] + [n]([r_1s_2] + [r_2s_1]) \\
                                          &= [r_1r_2] - [s_1s_2] + [n][r_1s_2] + [n][r_2s_1] \\
\end{align*}
Now, recall that since $q = n^2 + 1$, we have that $n^2 \equiv -1 \ (mod \ q)$, so $[n^2] = [-1]$. Hence:
\begin{align*}
    \phi((r_1 + \i s_1))\phi((r_2 + \i s_2)) &= ([r_1] + [n][s_1])([r_2] + [n][s_2]) \\
                                             &= [r_1][r_2] + [n][r_1][s_2] +   [n][r_2][s_1] + [n][n][s_1][s_2]  \\
                                             &= [r_1r_2] + [n][r_1s_2] +   [n][r_2s_1] + [n^2][s_1s_2]  \\
                                             &= [r_1r_2] + [-1][s_1s_2]  + [n][r_1s_2] +   [n][r_2s_1]  \\
                                             &= [r_1r_2] - [1][s_1s_2]  + [n][r_1s_2] +   [n][r_2s_1]  \\
                                             &= [r_1r_2] - [s_1s_2]  + [n][r_1s_2] +   [n][r_2s_1]  \\
\end{align*}
Hence $\phi((r_1 + \i s_1)(r_2 + \i s_2)) = \phi((r_1 + \i s_1))\phi((r_2 + \i s_2))$.

\smallskip

Thus, $\phi$ satisfies the properties of a homormorphism. Moreover, it is surjective. Indeed, pick any $[x] \in \mathbb{Q}$. Then:
\[
\phi(x + 0\i) = [x] + [n][0] = [x]
\]
Hence, for any element in $\mathbb{Z}_q$, $\phi$ maps at least one element of $\mathbb{Z}[\i]$ to it, so $\phi$ is surjective. 

\question \textbf{Determine $ker(\phi)$.}

$ker(\phi)$ is the set of all $r + \i s \in \mathbb{Z}[\i]$ with:
\[
\phi(r + \i s) = [r] + [n][s] = [r + ns] = [0] \ \iff \ r + ns \equiv 0 \ (mod \ q) \ \iff \ r + ns = kq, k \in \mathbb{Z}
\]
In other words, we require that $r = kq - ns$. Hence:
\[
ker(\phi) = \{kq - ns + \i s \ | \ k \in \mathbb{Z}, s \in \mathbb{Z}\}
\]

\question \textbf{Prove that the quotient ring $\mathbb{Z}[\i]/I_n$ is a field fi and only if $q = n^2 + 1$ is prime.}

We claim that $I_n = ker(\phi)$. Firstly, notice that $q \in I_n$. $I_n$ is the set of all Gaussian integers which have $\i - n$ as a factor:
\[
I_n = \{m(\i - n) \ | \ m \in \mathbb{Z}[\i]\}
\]
Notice, $q = n^2 + 1 \in I_n$, since:
\[
n^2 + 1 = (\i -n)(-\i-n)
\]
Because of this, $ker(\phi) \subseteq I_n$, since:
\[
ker(\phi) = \{kq - ns + \i s \ | \ k \in \mathbb{Z}, s \in \mathbb{Z}\}
\]
and:
\[
kq - ns + \i s = k(\i -n)(-\i-n) + s(\i -n) = (\i - n)(s - k(\i + n)) \in I_n
\]
since $s - k(\i + n) = (s-kn) - \i k \in \mathbb{Z}[\i]$.

\smallskip

Now consider some $x \in I_n$. It must have the form:
\[
x = (a + \i b)(\i - n), \qquad a,b \in \mathbb{Z}
\]
But now, it must be the case that $x \in ker(\phi)$, since any element in $ker(\phi)$ can be written as:
\[
((s-kn) - \i k)(\i - n), \qquad k,s \in \mathbb{Z}
\]
so setting $k = -b$ and $s = a - bn$ ensures that:
\[
((s-kn) - \i k)(\i - n) = ((a - bn + bn) + \i b)(\i - n) = (a + \i b)(\i - n) = x
\]
Hence, it must be the case that $I_n \subseteq ker(\phi)$.

\smallskip

Thus, since $ker(\phi) \subseteq I_n$ and $I_n \subseteq ker(\phi)$, we must have that $I_n = ker(\phi)$.

\smallskip

We have that $\phi : \mathbb{Z}[\i] \to \mathbb{Z}_q$ is a surjective ring homomorphism, with $ker(\phi) = I_n$. Thus, by the first isomorphism theorem for rings, it must be the case that there exists a ring isomorphism:
\[
\bar{f} : \mathbb{Z}[\i]]/I_n \to \mathbb{Z}_q
\]
so in particular $\mathbb{Z}[\i]/I_n$ is isomorphic to $\mathbb{Z}_q$.

\smallskip

Now, recall that $\mathbb{Z}_q$ is a field \textbf{if and only if} $q = n^2 + 1$ is prime (Proposition 3.1.11). Moreover, recall that a field is a non-zero, commutative ring in which every non-zero element is a unit.

\smallskip

$\mathbb{Z}[\i]/I_n$ is clearly non-zero, since for example:
\[
E(1) = \{1 + i \ | \ i \in I_n\}
\]
is an element of $\mathbb{Z}[\i]/I_n$.

\smallskip

Moreover, $\mathbb{Z}[\i]/I_n$ is a ring by definition, and in fact it is a commutative ring. Let $x,y \in \mathbb{Z}[\i]/I_n$. Consider:
\[
\bar{f}(xy) = \bar{f}(x)\bar{f}(y) = \bar{f}(y)\bar{f}(x) = \bar{f}(yx)
\]
where we have used the fact that $\mathbb{Z}_q$ is a commutative ring. Since $\bar{f}$ is an isomorphism, it is bijective (so in particular injective), so $\bar{f}(xy) = \bar{f}(yx)$ implies that $xy = yx$, so $\mathbb{Z}[\i]/I_n$ is a commutative ring.

\smallskip

Lastly, we show that all the elements of $\mathbb{Z}[\i]/I_n$ are units if and only if $q$ is prime. Before this, consider the following:

\textbox{
We claim that if we have a ring isomorphism $g : A \to B$, then $g(1_A) = 1_B$. Since $g$ is surjective, $\exists a \in A$ such that $g(a) = 1_B$ (and bijectivity tells us this $a$ is in fact unique). Then:
\[
1_B = g(a) = g(1_A \cdot a) = g(1_A)g(a) = g(1_A) \cdot 1_B = g(1_A) 
\]
So in particular, $g(1_A) = 1_B$. Since $g$ is isomorphic, we have that $a = 1_A$ is the only element in $A$ which maps to $1_B$.
[Lemma 1]
}

Now, let $R = \mathbb{Z}[\i]/I_n$ and let $S = \mathbb{Z}_q$. Firstly, assume that $q$ is not prime, and that $\mathbb{Z}[\i]/I_n$ only contains units. Then, pick $x \in \mathbb{Z}[\i]/I_n, x \neq 0_R$ such that $\bar{f}(x)$ is not a unit (since $q$ is not prime, $\mathbb{Z}_q$ is not a field, and $\bar{f}$ is bijective, so such an $x$ must exist). Then, $\exists y \in \mathbb{Z}[\i]/I_n$ such that:
\[
xy = 1_R \ \implies \ \bar{f}(xy) = \bar{f}(x)\bar{f}(y) = \bar{f}(1_R)
\]
By Lemma 1, it must be the case that $f(1_R) = 1_S$. But then, this would imply that $\bar{f}(x)\bar{f}(y) = 1_S$, which is a contradiction, since $\bar{f}(x)$ is not a unit. Hence, if $q$ is not prime, $R = \mathbb{Z}[\i]/I_n$ is not fully composed of units, so in particular, it can't be a field.

\smallskip

Alternatively, consider the case in which $q$ is prime, so $\mathbb{Z}_q$ is a field. Pick any non-zero element $x \in R$, and define $y = \bar{f}(x) \in \mathbb{Z}_q$. Since $\mathbb{q}$ is a field, we know that $y^{-1} \in \mathbb{Z}_q$ exists. Moreover, $\bar{f}$ is an isomorphism, so its inverse $\bar{f}^{-1}$ exists. Then:
\begin{align*}
    \bar{f}(x \cdot \bar{f}^{-1}(y^{-1})) &= \bar{f}(x)\bar{f}(\bar{f}^{-1}(y^{-1})) \\
                                   &= yy^{-1} \\
                                   &= 1_S \\
\end{align*}
By Lemma 1:
\[
\bar{f}(x\bar{f}^{-1}(y^{-1})) = 1_S \ \iff \ x\bar{f}^{-1}(y^{-1}) = 1_R
\]
In other words, $x$ has an inverse, and so it is a unit. In particular, since $x$ was an arbitrary element of $\mathbb{Z}[\i]/I_n$, it means that every element of $\mathbb{Z}[\i]/I_n$ is a unit when $q$ is prime.

\smallskip

Thus, we have shown that $\mathbb{Z}[\i]/I_n$ is a non-empty commutative ring, with all its elements being units, if and only if $q$ is prime.



\end{questions}

\end{document}