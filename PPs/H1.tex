\documentclass{exam}
\usepackage[utf8]{inputenc}

\usepackage[bottom = 2cm, left = 1.5cm, right = 1.5cm]{geometry}

\usepackage{mynotes}

\title{Honours Algebra - Practicing Proof 1}
\author{s1908368}
\date{February 2022}

\begin{document}

\maketitle

Let $V$ be a vector space, over a field $F$, of dimension $r$.

\begin{questions}

\question \textbf{Show that $W := Hom_F(V,F)$ is also a vector space over $F$.}

\bigskip

Define $W$ to be the vector space of homomorphisms of the form $p : V \to F$, where:
\[
(f+g)(x) = f(x) + g(x), \qquad f,g \in W, x \in V
\]
\[
(\lambda f)(x) = \lambda (f(x)), \qquad f \in W, x \in V, \lambda \in F
\]

Notice, since $f(x), g(x) \in F$ then $f(x) + g(x) \in F$. Similarly, since $f(x), \lambda \in F$, then $\lambda(f(x)) \in F$ too. Thus, this definition for elements in $W$ satsifies the closure under addition and scalar multiplication.

\bigskip

For $W$ to be a vector space, it must satisfy the 2 distributive laws and the associative law. It must also contain an identity element.  Let $f,g \in W$, let $x \in V$, let $\lambda, \mu \in F$.

\bigskip

The first distributive law states that:
\[
(\lambda(f + g))(x) = (\lambda f + \lambda g)(x)
\]
Indeed:
\begin{align*}
    (\lambda(f + g))(x) &= \lambda (f + g)(x) \\
                        &= \lambda (f(x) + g(x)) \\
                        &= \lambda (f(x)) + \lambda(g(x)) \\
                        &= (\lambda f)(x) + (\lambda g)(x) \\
                        &= (\lambda f + \lambda g)(x) \\
\end{align*}

The second distributive law states that:
\[
((\lambda + \mu)f)(x) = (\lambda f + \mu f)(x)
\]
Indeed:
\begin{align*}
    ((\lambda + \mu)f)(x) &= (\lambda + \mu)f(x) \\
                        &= \lambda (f(x)) + \mu (f(x)) \\
                        &= (\lambda f)(x) + (\mu f)(x) \\
                        &= (\lambda f + \mu f)(x) \\
\end{align*}

The associative law states that:
\[
(\lambda(\mu f))(x) = (\mu(\lambda f))(x)
\]
Indeed:
\begin{align*}
    (\lambda(\mu f))(x) &= \lambda (\mu f)(x) \\
                        &= \lambda (\mu f)(x) \\
                        &= (\lambda \mu) f(x) \\
                        &= (\mu \lambda) f(x) \\
                        &= \mu (\lambda f)(x) \\
                        &= (\mu (\lambda f))(x) \\
\end{align*}
For the identity element, we can just pick $1_F$, since then:
\[
1_Ff(x) = f(x)
\]

\question \textbf{Show that the dimension of $W$ is $r$.}

\bigskip

For this\footnote{This theorem is similar to 1.7.7, but instead of using $F^n$, we employ generic vector spaces.}, we show that for vector spaces $G,H$ if there exists an isomorphism $\phi : G \to H$, then $dim G = dim H$. To do this, let $A = \{v_1, \ldots, v_n\}$ be a basis for $G$. We show that $B = \{\phi(v_1), \ldots, \phi(v_n)\}$ is a basis of $H$. If this is the case, then $|A| = |B|$ so $dim G = dim H$.

\smallskip

$B$ is linearly independent, since for $\alpha_i \in F, i \in [1,n]$:
\[
\sum_{i = 1}^n \alpha_i \phi(v_i) = 0 \ \implies \ \phi\left(\sum_{i = 1}^n \alpha_i v_i\right) = 0
\]
where we have used the lienarity of the isomorphism. But notice, since $\phi(0) = 0$, and $\phi$ is injective, we must have that:
\[
\sum_{i = 1}^n \alpha_i v_i = 0
\]
But since $A$ is a basis, the above is true only with $\alpha_i = 0, \forall i \in [1,n]$. In other words, the set $B$ is linearly independent.

\smallskip

$B$ also generates $H$. Pick any $h \in H$. Since $\phi$ is surjective, $\exists g \in G$ such that:
\[
\phi(g) = h
\]
Since $g \in G$, we can express it as a linear combination of elements in $A$:
\[
g = \sum_{i = 1}^n \alpha_i v_i, \qquad \alpha_i \in F, i \in [1,n]
\]
Applying $\phi$ to the above, and exploiting its linearity:
\[
\phi(g) = \phi\left(\sum_{i = 1}^n \alpha_i v_1\right) \ \implies \ h = \sum_{i = 1}^n \alpha_i \phi(v_i)
\]
In other words, $B$ is also generating. Hence, $B$ must be a basis of $H$. Since $|A| = |B|$ it follows that $dim G = dim H$, as required.

\bigskip

Now recall Theorem 2.3.1 of the notes, which states that exists an isomorphism of the form:
\[
Hom_F(G,H) \to Mat(n \times m; F)
\]
where $dim G = m$ and $dim H = n$. Applying this to the question, it follows that we must have an isomorphism:
\[
W \to Mat(r \times 1; F)
\]
where we are using the fact that the dimension of $F$ over itself is 1 (since the multiplicative identity of $F$ can be used as the only element of its basis). By the Theorem we just proved, it thus follows that $dim W = dim Mat(r \times 1; F)$. But notice a basis for $Mat(r \times 1; F)$ can be constructed by using $r$ matrices, where each matrix in the basis $A_i$ is 0 in all entries except the $i$th entry (since elements in $Mat(r \times 1; F)$ have a single column, they can be thought of as a column vector - in fact, this basis corresponds to the standard basis for $F^r$). Hence, it follows that:
\[
dim Mat(r \times 1; F) = r \ \implies \ dim W = r
\]
as required.

\bigskip

\question \textbf{Show that $U := Hom_F(W,F)$ is also a vector space over $F$, of dimension $r$.}

\bigskip

Since in question 1 we described $V$ as a generic vector space of dimension $r$, it follows that the results obtained in parts 1 and 2 apply to $U$ aswell, since it is a vector space of all homomorphism from a vector space of dimension $r$, and the field over which the vector space is defined.

By question 1, defining operation in $U$ identically to how they were defined for $W$ (but with $f,g \in U,$ and $f : W \to F, g : W \to F$), it follows that $U$ must be a vector space over $F$.

By question 2, it must be the case that $dim U = r$ (since we can construct an isomorphism from $U$ to $Mat(r \times 1; F)$, and $dim Mat(r \times 1; F) = r$ so $dim U = r$).

\bigskip

\question \textbf{Finally, construct an isomorphism $\phi : V \to U$ without choosing a basis of $V$.}

\bigskip

By example 1.7.4 of the notes, we know that there exists an isomorphism:
\[
\Phi : F^r \to X 
\]
where $X$ is a vector space over the field $F$, and $dim_F(X) = r$.

Applying them to $V,U$, define isomorphisms:
\[
\phi_V : F^r \to V \qquad \phi_U : F^r \to U
\]
Notice, $\phi_V^{-1} : V \to F^r$ is the inverse of $\phi_V$, and it is also an isomorphism. Then define:
\[
\phi := \phi_U \circ \phi_V^{-1} \qquad \phi : V \to U
\]
$\phi$ is defined as a composition of isomorphisms, and so it is an isomorphism (since composition of bijections produces a bijection, and composition of homomorphisms produces a homomorphism).


\end{questions}

\end{document}